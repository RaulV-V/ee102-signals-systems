% The following packages can be found on http:\\www.ctan.org
% \usepackage{graphics} % for pdf, bitmapped graphics files
%\usepackage{epsfig} % for postscript graphics files
%\usepackage{mathptmx} % assumes new font selection scheme installed
%\usepackage{times} % assumes new font selection scheme installed
\usepackage{amsmath} % assumes amsmath package installed
\usepackage{amssymb,mathtools}  % assumes amsmath package installed
\usepackage{xcolor}
\usepackage{pgfplots,subcaption}
\usepackage[hidelinks]{hyperref}
\usepackage{verbatim}
\usepackage{graphicx}
\usepackage{listings}
\usepackage{fancyhdr}
\usepackage{siunitx}
\usepackage[most]{tcolorbox}
\usepackage{enumitem}
% -------- listings (Python) ----------
\lstdefinestyle{py}{
  language=Python,
  basicstyle=\ttfamily\small,
  keywordstyle=\color{blue!60!black}\bfseries,
  commentstyle=\color{green!40!black},
  stringstyle=\color{orange!60!black},
  showstringspaces=false,
  columns=fullflexible,
  frame=single,
  framerule=0.3pt,
  numbers=left,
  numberstyle=\tiny,
  xleftmargin=1em,
  tabsize=2,
  breaklines=true,
}
\usepackage[american]{circuitikz}
\usepackage{tikz}
\usepackage{caption}    
\usepackage{lscape}
\usepackage{soul}
\usepackage{tikz}
\usepackage{physics}
\usetikzlibrary{calc,angles,quotes,arrows.meta}

\usepackage{hyperref}
\hypersetup{
    colorlinks=true,
    linkcolor=blue,
    filecolor=magenta,      
    urlcolor=blue,
    pdftitle={week1_notes},
    pdfpagemode=FullScreen,
}
%\usepackage{float} 

%\usepackage[demo]{graphicx}
\pgfplotsset{compat=1.18}
% \usepgfplotslibrary{fillbetween}

\newsavebox{\measurebox}

\let\proof\relax\let\endproof\relax


\def\abs#1{\left\lvert#1\right\rvert}
\let\proof\relax
\let\endproof\relax
\usepackage{amsthm}
\usepackage{accents}
\usepackage{relsize}
\newcommand{\ubar}[1]{\underaccent{\bar}{#1}}
\newtheorem{theorem}{Theorem}
\newtheorem{corollary}{Corollary}[theorem]
\newtheorem{lemma}{Lemma}
\newtheorem{proposition}{Proposition}
\newtheorem{statement}{Statement}

\theoremstyle{definition}
\newtheorem{definition}{Definition}
 
\theoremstyle{remark}
\newtheorem*{remark}{Remark}
\theoremstyle{remark}
\newtheorem*{claim}{Claim}
\setlength{\parindent}{0cm}
\newenvironment{nalign}{
    \begin{equation}
    \begin{aligned}
}{
    \end{aligned}
    \end{equation}
    \ignorespacesafterend
}
% --- Pop-Quiz environment + solutions (no 'environ' needed) ------------
% Numbering: section-based; change to global by replacing the definition of \thepopquiz
\newcounter{popquiz}
\renewcommand{\thepopquiz}{\thesection.\arabic{popquiz}} % or: \renewcommand{\thepopquiz}{\arabic{popquiz}}

% Storage + flag
\makeatletter
\newcommand\popquizsolutions{}     % holds all printed solutions
\newif\ifpopquizsolutionsflag
\popquizsolutionsflagfalse
\newcommand\currentpq{}            % holds current quiz label (snapshotted on begin)

% Pop-quiz box using tcolorbox' counter integration
\newtcolorbox[use counter=popquiz,number within=section]{popquiz}[1][]{%
  enhanced,breakable,
  colback=gray!10, colframe=black,
  title={Pop Quiz \thepopquiz}, fonttitle=\bfseries,
  before upper=\global\edef\currentpq{\thepopquiz}, % snapshot label for solutions
  #1
}

% Command to record a solution (optionally override the label)
\newcommand{\popquizsolution}[2][]{%
  \popquizsolutionsflagtrue
  \def\pq@label{#1}%
  \ifx\pq@label\@empty \edef\pq@label{\currentpq}\fi
  \begingroup
    \protected@edef\pq@tmp{%
      \noexpand\par\noexpand\noindent
      \noexpand\textbf{Pop Quiz \pq@label\noexpand\ Solution. }%
      \unexpanded{#2}%
      \noexpand\par
    }%
  \endgroup
  \g@addto@macro\popquizsolutions{\pq@tmp}%
}

% Print all collected solutions at the end (only if any were added)
\AtEndDocument{%
  \ifpopquizsolutionsflag
    \clearpage
    \section*{Pop Quiz Solutions}
    \popquizsolutions
  \fi
}
\makeatother
