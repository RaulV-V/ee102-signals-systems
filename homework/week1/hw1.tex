\makeatletter
\def\input@path{{../styles/}{../../styles/}{../../../styles/}{../}{../../}{../../../}}
\makeatother


\documentclass{ee102_pset}
% macros.tex - Course meta information
\renewcommand{\course}{EE 102}
\renewcommand{\coursetitle}{Signal Processing and Linear Systems}
\renewcommand{\instructor}{Ayush Pandey}
\renewcommand{\semester}{Fall}
\renewcommand{\year}{2025}
\renewcommand{\shorttitle}{Week 1: Introduction to Signals}
% Use \renewcommand to avoid 'already defined' errors

\usepackage{amsmath}
\usepackage{amssymb}
\usepackage{graphicx}
\usepackage{listings}
\usepackage[american]{circuitikz}

% Assignment info
\author{\rule{3cm}{0.4pt}} % Name placeholder
\submitdate{\rule{3cm}{0.4pt}} % Submission date placeholder
\problemset{Homework \#1: Introduction to Signals}
\renewcommand{\duedate}{September 7, 2025}
\shorttitle{Homework \#1}

\begin{document}

% Problem 1
\problem{1}

[Adapted from Problem 3 in Vierinen and Jensen] A time scaling system adjusts the scaling of the independent variable: $y(t) = x(\alpha t)$, when $x(t)$ is the signal fed into the system
and $y(t)$ is the output. Answer the following questions for this system:
\begin{enumerate}
    \item \textbf{[5 points]} Draw a block diagram for the system. Label the input, output, and the system.
    \item \textbf{[5 points]} What is the effect on the signal when $0 < \alpha < 1$. What about $\alpha > 1$?
    
    \item \textbf{[5 points]} Prove that the system is linear.
    \item \textbf{[5 points]} Give an example of a real-world application where studying this system can be useful. Then, for this example, propose a nonlinear modification to the system, which captures a real situation.  
\end{enumerate}

\vspace*{\fill}
\begin{center}
[use more pages if needed]
\end{center}
\problem{2}
Consider the signal $x(t) = a^{-tu(t)}$, where $u(t)$ is the unit step function. 
\begin{enumerate}
    \item \textbf{[5 points]} Sketch $x(t)$ for time $-2 < t < 2$ for $a > 0$. You are not allowed to use computer programs to do this.
    \item \textbf{[5 points]} Sketch $y(t)$ for time $-2 < t < 2$ for $a>0$, where $y(t) = 2x(5 - 0.5t)$
    \item \textbf{[5 points]} Find out whether the signal $y(t)$ is time-invariant.
    \item \textbf{[5 points]} Find out whether the signal $y(t)$ converges to 0 as $t \to \infty$ for $a > 0$. 
    \item \textbf{[5 points]} Find the value of $a$ such that $y(1) = 0.1$.
    \item \textbf{[5 points]} Think of a signal that you can relate $x(t)$ with. Analyze the properties of $x(t)$ to find out a real-world signal that shares similar characteristics.
\end{enumerate}
\vspace*{\fill}
\begin{center}
[use more pages if needed]
\end{center}

\problem{3}
For each of the signals below, you have four tasks:
\begin{enumerate}
  \item \textbf{[2 points, per signal]} Sketch the signal (clearly label the amplitude and axes)
  \item \textbf{[8 points, per signal]} Compute \(E_{\infty}\) and \(P_{\infty}\) using the definitions provided in the lecture notes.
  \item \textbf{[3 points, per signal]} Using Python (or MATLAB), plot the signal over an appropriate interval and confirm your findings.
  \item \textbf{[3 points total]} Briefly discuss a relevant example where the properties of the signal can be important.
\end{enumerate}
You must do the three parts above for each of the three signals below
\begin{enumerate}
\item[(a)] $x_1[n] = \left(\tfrac{1}{3}\right)^{n} u[n]$

\item[(b)] $x_2(t) = \mathrm{e}^{\mathrm{j}\,(3t+\pi/7)}$

\item[(c)] $x_3[n] = \mathrm{e}^{\mathrm{j}\,(\tfrac{\pi}{3}n+\pi/10)}$
\end{enumerate}
\textit{Hint:} To sketch a signal, compute its values for multiple values of time to understand the pattern. Remember that $u(n)$ is the continuous-time unit step function and $u[n]$ is the discrete-time unit step function. For signals in complex polar form, you can sketch the real part and the imaginary part on separate axes.  
\vspace*{\fill}
\begin{center}
[use more pages if needed]
\end{center}

\problem{4} Please write a short reflection on how many hours you dedicated to completing this homework assignment (do not include the time spent on the pre-requisite assignment). Also, please rate the difficulty of this assignment according to you. \textbf{[2 points]}
\end{document}