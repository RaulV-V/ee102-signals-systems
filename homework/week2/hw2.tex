\makeatletter
\def\input@path{{../styles/}{../../styles/}{../../../styles/}{../}{../../}{../../../}}
\makeatother


\documentclass{ee102_pset}
% macros.tex - Course meta information
\renewcommand{\course}{EE 102}
\renewcommand{\coursetitle}{Signal Processing and Linear Systems}
\renewcommand{\instructor}{Ayush Pandey}
\renewcommand{\semester}{Fall}
\renewcommand{\year}{2025}
\renewcommand{\shorttitle}{Week 1: Introduction to Signals}
% Use \renewcommand to avoid 'already defined' errors

% The following packages can be found on http:\\www.ctan.org
% \usepackage{graphics} % for pdf, bitmapped graphics files
%\usepackage{epsfig} % for postscript graphics files
%\usepackage{mathptmx} % assumes new font selection scheme installed
%\usepackage{times} % assumes new font selection scheme installed
\usepackage{amsmath} % assumes amsmath package installed
\usepackage{amssymb,mathtools}  % assumes amsmath package installed
\usepackage{xcolor}
\usepackage{pgfplots,subcaption}
\usepackage[hidelinks]{hyperref}
\usepackage{verbatim}
\usepackage{graphicx}
\usepackage{listings}
\usepackage{fancyhdr}
% \usepackage{geometry}
\usepackage{siunitx}
\usepackage[most]{tcolorbox}
\usepackage{enumitem}
\usepackage{environ}
% -------- listings (Python) ----------
\lstdefinestyle{py}{
  language=Python,
  basicstyle=\ttfamily\small,
  keywordstyle=\color{blue!60!black}\bfseries,
  commentstyle=\color{green!40!black},
  stringstyle=\color{orange!60!black},
  showstringspaces=false,
  columns=fullflexible,
  frame=single,
  framerule=0.3pt,
  numbers=left,
  numberstyle=\tiny,
  xleftmargin=1em,
  tabsize=2,
  breaklines=true,
}

\usepackage[american]{circuitikz}
\usepackage{tikz}
\usetikzlibrary{arrows.meta,positioning,calc,angles,quotes}
\tikzset{
  >={Latex[length=2.2mm]},
  block/.style={draw, thick, rectangle, minimum height=10mm, minimum width=24mm, align=center},
  gain/.style={block, minimum width=14mm},
  sum/.style={draw, thick, circle, inner sep=0pt, minimum size=6mm},
  conn/.style={-Latex, thick},
}
\usepackage{caption}    
\usepackage{lscape}
\usepackage{soul}
\usepackage{physics}
\usepackage{hyperref}
\hypersetup{
    colorlinks=true,
    linkcolor=blue,
    filecolor=magenta,      
    urlcolor=blue,
    pdftitle={week1_notes},
    pdfpagemode=FullScreen,
}
%\usepackage{float} 

%\usepackage[demo]{graphicx}
\pgfplotsset{compat=1.18}
% \usepgfplotslibrary{fillbetween}

\newsavebox{\measurebox}

\let\proof\relax\let\endproof\relax


\def\abs#1{\left\lvert#1\right\rvert}
\let\proof\relax
\let\endproof\relax
\usepackage{amsthm}
\usepackage{accents}
\usepackage{relsize}
\newcommand{\ubar}[1]{\underaccent{\bar}{#1}}
\newtheorem{theorem}{Theorem}
\newtheorem{corollary}{Corollary}[theorem]
\newtheorem{lemma}{Lemma}
\newtheorem{proposition}{Proposition}
\newtheorem{statement}{Statement}

\theoremstyle{definition}
\newtheorem{definition}{Definition}
 
\theoremstyle{remark}
\newtheorem*{remark}{Remark}
\theoremstyle{remark}
\newtheorem*{claim}{Claim}
\setlength{\parindent}{0cm}
\newenvironment{nalign}{
    \begin{equation}
    \begin{aligned}
}{
    \end{aligned}
    \end{equation}
    \ignorespacesafterend
}

\usepackage{amsmath}
\usepackage{amssymb}
\usepackage{graphicx}
\usepackage{listings}
\usepackage[american]{circuitikz}

% Assignment info
\author{\rule{3cm}{0.4pt}} % Name placeholder
\submitdate{\rule{3cm}{0.4pt}} % Submission date placeholder
\problemset{Homework \#2: Properties of Signals}
\renewcommand{\duedate}{September 14, 2025}
\shorttitle{Homework \#2}

\begin{document}

% Problem 1
\problem{1}
[Adapted from Lathi 1.1-11] Consider a signal that is a sum of complex exponentials given by
\[
x(t)=\sum_{k=m}^{n} D_k\,e^{j\omega_k t}\!
\]
where $D_k\in\mathbb{C}$ and $\{\omega_k\}$ are pairwise distinct, that is, $\omega_i \neq \omega_j$ for all $i \neq j$.

\problempart{[5 points]} Show that the time-averaged power of $x(t)$ is
  \[
  P_\infty(x)=\lim_{T\to\infty}\frac{1}{2T}\int_{-T}^{T}|x(t)|^2\,dt
  \;=\;\sum_{k=m}^{n}|D_k|^2.
  \]
  \emph{Hint:} expand $|x(t)|^2$ and use that $\frac{1}{2T}\!\int_{-T}^{T}\!e^{j(\omega_k-\omega_\ell)t}\,dt\to 0$ when $\omega_k\neq\omega_\ell$.

\problempart{[5 points]} Determine $E_\infty(x)=\int_{-\infty}^{\infty}|x(t)|^2\,dt$. State clearly whether it is finite or infinite for this $x(t)$ and justify.

\problempart{[5 points]} Let
  \[
  \tilde{x}(t)\triangleq \overline{x(t)}=\sum_{k=m}^{n}\overline{D_k}\,e^{-j\omega_k t}.
  \] What is the power for this conjugate signal?

\problempart{[5 points]} Is $x(t)$ even, odd, or neither? If it is not necessarily even/odd, state the sufficient conditions on $\{D_k,\omega_k\}$ under which $x(t)$ becomes even or odd. For a challenge, you can attempt to show both necessary and sufficient conditions.
\vspace*{\fill}
\begin{center}
[use more pages if needed]
\end{center}


\problem{2} Use the unit step $u(t)$ and unit impulse $\delta(t)$ function definitions discussed in class to answer the following questions.
\problempart{[5 points]} Show that
  \[
  \int_{-\infty}^{t} u(\tau)\,d\tau = t\,u(t) \quad (\text{the ramp } r(t)).
  \]

\problempart{[5 points]} Show that
  \[
  \int_{-\infty}^{t}\delta(\tau)\,d\tau = u(t).
  \]

\problempart{[5 points]} Prove in the sense of distributions that
  \[
  \frac{d}{dt}\,u(t)=\delta(t),\qquad \frac{d}{dt}\,[t\,u(t)] = u(t) + t\,\delta(t).
  \]

% \problempart{[5 points]} Let
%   \[
%   \delta_\varepsilon(t)\triangleq \frac{1}{\varepsilon}\,\mathrm{rect}\!\left(\frac{t}{\varepsilon}\right),
%   \quad \text{so that}\quad \int_{-\infty}^{\infty}\delta_\varepsilon(t)\,dt=1.
%   \]
%   Compute $E_\infty(\delta_\varepsilon)=\int_{-\infty}^{\infty}|\delta_\varepsilon(t)|^2\,dt$ and evaluate $\lim_{\varepsilon\to 0^+}E_\infty(\delta_\varepsilon)$. Interpret your result for the unit impulse $\delta$.

\problempart{[5 points]}  For the finite pulse
  \[
  p(t)=A\,\mathrm{rect}\!\left(\frac{t-t_0}{T}\right),
  \]
  compute $E_\infty(p)$ and $P_\infty(p)$ and state when each is finite/nonzero.

\vspace*{\fill}
\begin{center}
[use more pages if needed]
\end{center}

\problem{3}
For each signal below, state whether it is periodic. If periodic, find the fundamental period ($T_0$ for continuous time, $N_0$ for discrete time). If it is not periodic, justify your answer.

\problempart{[5 points]} $x(t)=5\sin(10t-0.5)+\cos(5t)$
\problempart{[5 points]} $x(t)=j\,e^{j10t}$
\problempart{[5 points]} $x(t)=e^{\,(-0.5+j)\,(t+0.5)}$
\problempart{[5 points]} $x[n]=e^{j\,12\pi n}$
\problempart{[5 points]} $x[n]=1+e^{j\,\frac{4\pi}{7}n}-e^{j\,\frac{2\pi}{3}n}$

\vspace*{\fill}
\begin{center}
[use more pages if needed]
\end{center}

\problem{4 } [Adapted from Vierinen Ch.5 P7] Practice with real audio signals. The file \texttt{amplifier.ipynb} (on \href{https://github.com/ee-ucmerced/ee102-signals-systems/tree/main/homework/week2/amplifier.ipynb}{GitHub}) implements the linear amplifier system. The amplified signal is
\[
y(t)=\alpha\,x(t).
\]
The code reads \texttt{guitar\_clean.wav} (file on \href{https://github.com/ee-ucmerced/ee102-signals-systems/tree/main/homework/week2/guitar\_clean.wav}{GitHub}), plots original vs.\ amplified, normalizes the output to $0.9$ peak, and writes \texttt{guitar\_amp.wav}.

\problempart{[5 points]} Run the script, and produce a figure showing the original $x(t)$ and $\alpha x(t)$ on the same axes. Explain why the saved WAV (after peak normalization) does \emph{not} sound louder even though the plot shows amplification.

\problempart{[5 points]} Using the discrete-time samples $x[n]$ from the WAV file, estimate $P_\infty(x)$ and $P_\infty(\alpha x)$ via
  \[
  \widehat{P}=\frac{1}{N}\sum_{n=0}^{N-1}|x[n]|^2.
  \]
  Verify the relationship between the two powers before the peak normalization step.

\problempart{[5 points]} Modify the script to \emph{hard-clip} the amplified signal to $[-1,1]$ before saving (no renormalization). Plot waveforms and mark clipped regions. Discuss how clipping affects the spectrum qualitatively.

\problempart{[5 points]} Is $y(t)=\alpha x(t)$ linear? time-invariant? Is the \emph{clipper} system that we discussed during lecture linear? time-invariant? Briefly justify each of the four properties of the systems.
\vspace*{\fill}
\begin{center}
[use more pages if needed]
\end{center}

\end{document}