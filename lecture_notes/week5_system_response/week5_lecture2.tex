\makeatletter
\def\input@path{{../styles/}{../../styles/}{../../../styles/}{../}{../../}{../../../}}
\makeatother
\documentclass{ee102_notes}
% macros.tex - Course meta information
\renewcommand{\course}{EE 102}
\renewcommand{\coursetitle}{Signal Processing and Linear Systems}
\renewcommand{\instructor}{Ayush Pandey}
\renewcommand{\semester}{Fall}
\renewcommand{\year}{2025}
\renewcommand{\shorttitle}{Week 1: Introduction to Signals}
% Use \renewcommand to avoid 'already defined' errors

% The following packages can be found on http:\\www.ctan.org
% \usepackage{graphics} % for pdf, bitmapped graphics files
%\usepackage{epsfig} % for postscript graphics files
%\usepackage{mathptmx} % assumes new font selection scheme installed
%\usepackage{times} % assumes new font selection scheme installed
\usepackage{amsmath} % assumes amsmath package installed
\usepackage{amssymb,mathtools}  % assumes amsmath package installed
\usepackage{xcolor}
\usepackage{pgfplots,subcaption}
\usepackage[hidelinks]{hyperref}
\usepackage{verbatim}
\usepackage{graphicx}
\usepackage{listings}
\usepackage{fancyhdr}
% \usepackage{geometry}
\usepackage{siunitx}
\usepackage[most]{tcolorbox}
\usepackage{enumitem}
\usepackage{environ}
% -------- listings (Python) ----------
\lstdefinestyle{py}{
  language=Python,
  basicstyle=\ttfamily\small,
  keywordstyle=\color{blue!60!black}\bfseries,
  commentstyle=\color{green!40!black},
  stringstyle=\color{orange!60!black},
  showstringspaces=false,
  columns=fullflexible,
  frame=single,
  framerule=0.3pt,
  numbers=left,
  numberstyle=\tiny,
  xleftmargin=1em,
  tabsize=2,
  breaklines=true,
}

\usepackage[american]{circuitikz}
\usepackage{tikz}
\usetikzlibrary{arrows.meta,positioning,calc,angles,quotes}
\tikzset{
  >={Latex[length=2.2mm]},
  block/.style={draw, thick, rectangle, minimum height=10mm, minimum width=24mm, align=center},
  gain/.style={block, minimum width=14mm},
  sum/.style={draw, thick, circle, inner sep=0pt, minimum size=6mm},
  conn/.style={-Latex, thick},
}
\usepackage{caption}    
\usepackage{lscape}
\usepackage{soul}
\usepackage{physics}
\usepackage{hyperref}
\hypersetup{
    colorlinks=true,
    linkcolor=blue,
    filecolor=magenta,      
    urlcolor=blue,
    pdftitle={week1_notes},
    pdfpagemode=FullScreen,
}
%\usepackage{float} 

%\usepackage[demo]{graphicx}
\pgfplotsset{compat=1.18}
% \usepgfplotslibrary{fillbetween}

\newsavebox{\measurebox}

\let\proof\relax\let\endproof\relax


\def\abs#1{\left\lvert#1\right\rvert}
\let\proof\relax
\let\endproof\relax
\usepackage{amsthm}
\usepackage{accents}
\usepackage{relsize}
\newcommand{\ubar}[1]{\underaccent{\bar}{#1}}
\newtheorem{theorem}{Theorem}
\newtheorem{corollary}{Corollary}[theorem]
\newtheorem{lemma}{Lemma}
\newtheorem{proposition}{Proposition}
\newtheorem{statement}{Statement}

\theoremstyle{definition}
\newtheorem{definition}{Definition}
 
\theoremstyle{remark}
\newtheorem*{remark}{Remark}
\theoremstyle{remark}
\newtheorem*{claim}{Claim}
\setlength{\parindent}{0cm}
\newenvironment{nalign}{
    \begin{equation}
    \begin{aligned}
}{
    \end{aligned}
    \end{equation}
    \ignorespacesafterend
}

\renewcommand{\releasedate}{September 29, 2025}

\newcommand{\Eblank}{\rule{3cm}{0.4pt}}
\newcommand{\Rankblank}{\rule{3cm}{0.4pt}}

\newcommand{\uof}[1]{u\!\left[#1\right]} % discrete-time unit step

\begin{document}

\section*{EE 102 Week 5, Lecture 2 (Fall 2025)}
\subsection*{Instructor: \instructor}
\subsection*{Date: \releasedate}

\section{Goals}
The main goal of this lecture is to learn how to visualize the process of convolution using graphs. 

\section{Review: Convolution definition}
Recall that in continuous-time, the output $y(t)$ of an LTI system with input $x(t)$ and impulse response $h(t)$ is given by the convolution integral:
\[
y(t) = x(t) * h(t) = \int_{-\infty}^{\infty} x(\tau) h(t - \tau) \, d\tau.
\]
\begin{popquiz}
Prove that convolution is commutative, i.e., show that $x(t) * h(t) = h(t) * x(t)$.
\popqsplit 
Write the convolution integral for left-hand side as 
\[
x(t) * h(t) = \int_{-\infty}^{\infty} x(\tau) h(t - \tau) \, d\tau.
\]
Now, let $s = t - \tau$. Then, $\tau = t - s$ and $d\tau = -ds$. When $\tau$ goes from $-\infty$ to $\infty$, $s$ goes from $\infty$ to $-\infty$. Thus, we can rewrite the integral as
\[
x(t) * h(t) = \int_{\infty}^{-\infty} x(t - s) h(s) (-ds) = \int_{-\infty}^{\infty} h(s) x(t - s) \, ds.
\]
This is exactly the convolution integral for $h(t) * x(t)$. Hence, convolution is commutative.
\end{popquiz}
\section{Discrete time convolution}
Similar to the derivation for continuous-time convolution, we can derive the discrete-time convolution sum. Consider a discrete-time LTI system with input $x[n]$, output $y[n]$, and impulse response $h[n]$. Note that for a discrete-time impulse $\delta[n]$, the output is $h[n]$. Recall the sifting property of the discrete-time impulse:
\[
x[n] = \sum_{k=-\infty}^{\infty} x[k] \delta[n - k].
\]
Using linearity and time-invariance of the system, we can write the output as
\[
y[n] = \sum_{k=-\infty}^{\infty} x[k] h[n - k].
\]
This is the discrete-time convolution sum, denoted by $y[n] = x[n] * h[n]$.

\section{Example: A discrete-time echo system}
An audio receiver system produces an echo. When excited by a unit impulse, it responds with an echo of magnitude $1$ at $n=0$ that decays exponentially as $\alpha^n$ for $\alpha \in (0,1)$ until $n=5$ (that is, for six seconds in total). You may assume that $\alpha = \tfrac{1}{2}$ for numerical parts. Answer the following:
\begin{enumerate}
    \item[(A)] Sketch the impulse response $h[n]$ and label $h[0], h[1], \ldots, h[5]$.
    \item[(B)] We want to understand the kind of echo that will be produced when the audio receiver system is excited by a pulse input of unit amplitude lasting three seconds, starting at $n=0$ and staying at unit amplitude until $n=3$. Find $y[n]$ for this input using convolution and show your steps.
\end{enumerate}
The impulse response of the system is
\[
h[n]=
\begin{cases}
\alpha^{n}, & 0\le n\le 5,\\
0, & \text{otherwise,}
\end{cases}
\]
The input is a unit amplitude tone that starts at $n=0$ and lasts three seconds. So, we can write the pulse signal for the input $x[n]$ as
\[
x[n]=u[n]-u[n-3]=
\begin{cases}
1, & n=0,1,2,\\
0, & \text{otherwise.}
\end{cases}
\]

Now, we can compute the output $y[n]$ using convolution:
\[
y[n]=\sum_{k=-\infty}^{\infty} x[k]\,h[n-k],
\]
and give $y[n]$ explicitly for all $n$ where it is nonzero. It is important that we are careful about all values of $n$ for which $y[n]$ is nonzero. Echos can last longer than the original sound!

% --- macro: draw x[k], h[n-k], and y[n] for a given n ---
\newcommand{\ConvPanel}[1]{%
  \def\n{#1}%
  \begin{tikzpicture}[x=0.8cm,y=2.2cm]

    % ===== Panel: x[k] =====
    \begin{scope}
      \draw[->] (\kmin-0.6,0) -- (\kmax+0.6,0) node[below] {$k$};
      \draw[->] (0,-0.02) -- (0,1.05);
      \node[above left] at (0,1.05) {$x[k]$};

      % stems (blue) with values on top; NO y-axis tick labels
      \foreach \k in {\kmin,...,\kmax}{
        \pgfmathparse{(\k>=0 && \k<=2) ? 1 : 0}
        \let\vx\pgfmathresult
        \ifdim \vx pt>0pt
          \draw[very thick,blue] (\k,0) -- (\k,\vx);
          \fill[blue] (\k,\vx) circle (1.4pt);
          \node[above=3pt] at (\k,\vx) {\scriptsize \pgfmathprintnumber[precision=1]{\vx}};
        \fi
        \node[below=1pt] at (\k,0) {\scriptsize \k};
      }
    \end{scope}

    % ===== Panel: h[n-k] =====
    \begin{scope}[shift={(0,-1.6)}]
      % FORCE axis to -6..10
      \draw[->] (-6.6,0) -- (10.6,0) node[below] {$k$};
      \draw[->] (0,-0.02) -- (0,1.05);
      \node[left] at (0,1.05) {$h[n-k],\ n=\n$};

      % stems (blue) with values on top; NO y-axis tick labels
      \foreach \k in {-6,...,10}{
        % support: 0 <= n-k <= HLen  <=> n-HLen <= k <= n
        \pgfmathparse{(\k>=\n-\HLen && \k<=\n) ? pow(\Alpha,\n-\k) : 0}
        \let\vhnk\pgfmathresult
        \ifdim \vhnk pt>0pt
          \draw[very thick,blue,opacity=0.95] (\k,0) -- (\k,\vhnk);
          \fill[blue] (\k,\vhnk) circle (1.4pt);
          \node[above=6pt] at (\k,\vhnk) {\tiny \pgfmathprintnumber[precision=4]{\vhnk}};
        \fi
        \node[below=3pt] at (\k,0) {\scriptsize \k};
      }
    \end{scope}

    % ===== Panel: y[n] =====
    \begin{scope}[shift={(0,-4.2)}]
      \draw[->] (\kmin-0.6,0) -- (\kmax+0.6,0) node[below] {$n$};
      \draw[->] (0,-0.02) -- (0,1.8);
      \node[right] at (0,1.8) {$y[n]$};

      % y[n] = h[n]+h[n-1]+h[n-2] with truncation to 0..HLen
      \pgfmathparse{(\n>=0   && \n<=\HLen)   ? pow(\Alpha,\n)   : 0} \let\yA\pgfmathresult
      \pgfmathparse{(\n-1>=0 && \n-1<=\HLen) ? pow(\Alpha,\n-1) : 0} \let\yB\pgfmathresult
      \pgfmathparse{(\n-2>=0 && \n-2<=\HLen) ? pow(\Alpha,\n-2) : 0} \let\yC\pgfmathresult
      \pgfmathparse{\yA+\yB+\yC} \let\yval\pgfmathresult

      % single stem (blue) with value on top; NO y-axis tick labels
      \draw[very thick,blue] (\n,0) -- (\n,\yval);
      \fill[blue] (\n,\yval) circle (1.8pt);
      \node[above=2pt] at (\n,\yval) {\scriptsize \pgfmathprintnumber[precision=2]{\yval}};

      \foreach \kk in {\kmin,...,\kmax} \node[below=3pt] at (\kk,0) {\scriptsize \kk};
    \end{scope}
  \end{tikzpicture}%
}


\subsection{Convolution computation (without visualizing)}

Let us compute the output for various values of $n$ using the convolution sum directly:
\[
\begin{array}{c|l}
n & y[n] \\ \hline
-2: & 0\\
-1: & 0\\
0: & 1\\
1: & 1+\alpha\\
2: & 1+\alpha+\alpha^2\\
3: & \alpha+\alpha^2+\alpha^3\\
4: & \alpha^2+\alpha^3+\alpha^4\\
5: & \alpha^3+\alpha^4+\alpha^5\\
6: & \alpha^4+\alpha^5+\alpha^6\\
7: & \alpha^5+\alpha^6\\
8: & \alpha^6\\
9: & 0\\
10: & 0
\end{array}
\quad\Rightarrow\quad
\text{with }\alpha=\tfrac12:\ 
y[0..8]=\Big[1,\tfrac32,\tfrac74,\tfrac78,\tfrac7{16},\tfrac7{32},\tfrac7{64},\tfrac3{64},\tfrac1{64}\Big].
\]
So, we find that the output $y[n]$ is nonzero for $n=0,1,\ldots,8$. In general, the output of convolution in discrete-time is equal to $N+M-1$ where $N$ and $M$ are the lengths of the two signals being convolved. Here, the length of $x[n]$ is $3$ and the length of $h[n]$ is $6$, so the length of $y[n]$ is $3+6-1=8$. \textbf{This is important!}
% send the rest of the content after this always at the end:
\subsection{Visualizing convolution (with graphs)}
Now, we will solve this by using illustrations of convolution. For each index $n=0,1,\dots $, draw three plots in a row for each $n$:
\[
x[k],\qquad h[n-k]\ \text{(as a function of $k$)},\qquad \text{and the resulting single sample }y[n],
\]
so that the overlap of $x[k]$ and $h[n-k]$ and the accumulation giving $y[n]$ are visually clear.

Let's start by drawing $h[n]$ for $\alpha = \tfrac{1}{2}$:

\def\Alpha{0.5}
\def\HLen{6}     % h[n] support: 0..6
\def\kmin{-2}    % plotting k-range
\def\kmax{10}

\begin{figure}[h]
    \centering
   
    \begin{tikzpicture}[x=0.8cm,y=4cm]
    \draw[->] (-0.8,0) -- (\HLen+1.2,0) node[below] {$n$};
    \draw[->] (0,-0.02) -- (0,1.1) node[left] {$h[n]$};
    \foreach \n in {0,...,\HLen}{
        \pgfmathparse{pow(\Alpha,\n)}
        \let\v\pgfmathresult
        \draw[very thick] (\n,0) -- (\n,\v);
        \fill (\n,\v) circle (1.6pt);
        \node[below=3pt] at (\n,0) {\small \n};
    }
    \end{tikzpicture}
    \label{fig:hn}
    \caption{Impulse response $h[n]$ for $\alpha=\tfrac12$.}
\end{figure}
% ---- h[k] figure (now k = -6..10), with y-axis ticks/labels and stem value labels ----
\subsection{Idea: Flip `h' and slide through `x'}
Note the x-axis labels carefully! We have $x[k]$ and $h[k]$ because we need these for the convolution sum. We are interested in finding $y[n]$ for each value of $n$. For each $n$, we have a $x[k]$ and $h[k]$ that we use for all values of $k$ to solve the convolution sum. Notice that $h[k]$ is not directly used in the convolution sum, instead we have $h[n-k]$. This means that for each value of $n$, we need to flip $h[k]$ around the vertical axis and then shift it by $n$ units to get $h[n-k]$.

For $n=0$, the convolution is visualized in Figure~\ref{fig:conv_n0}. 
\begin{figure}[h]
    \centering
    \ConvPanel{0}
    \caption{Convolution for $n=0$.}
    \label{fig:conv_n0}
\end{figure}


Then, for $n=1$, the convolution is visualized in Figure~\ref{fig:conv_n1} and for all other values of $n$ see Figures~\ref{fig:conv_n2},~\ref{fig:conv_n3},~\ref{fig:conv_n4},~\ref{fig:conv_n5},~\ref{fig:conv_n6},~\ref{fig:conv_n7},~\ref{fig:conv_n8}, and~\ref{fig:conv_n9}.

% \begin{center}\textbf{Sliding convolution panels}\end{center}
\begin{figure}[h]
    \centering
    \ConvPanel{1}
    \caption{Convolution for $n=1$.}
    \label{fig:conv_n1}
\end{figure}

\medskip
\begin{figure}[h]
    \centering
    \ConvPanel{2}
    \caption{Convolution for $n=2$.}
    \label{fig:conv_n2}
\end{figure}

\medskip

% All other values of $n$ are shown below:
\begin{figure}[h!]
    \centering
    \ConvPanel{3}
    \caption{Convolution for $n=3$.}
    \label{fig:conv_n3}
\end{figure}
\begin{figure}[h!]
    \centering
    \ConvPanel{4}
    \caption{Convolution for $n=4$.}
    \label{fig:conv_n4}
\end{figure}
\begin{figure}[h!]
    \centering
    \ConvPanel{5}
    \caption{Convolution for $n=5$.}
    \label{fig:conv_n5}
\end{figure}
\begin{figure}[h!]
    \centering
    \ConvPanel{6}
    \caption{Convolution for $n=6$.}
    \label{fig:conv_n6}
\end{figure}
\begin{figure}[h!]
    \centering
    \ConvPanel{7}
    \caption{Convolution for $n=7$.}
    \label{fig:conv_n7}
\end{figure}
\medskip
\begin{figure}[h!]
    \centering
    \ConvPanel{8}
    \caption{Convolution for $n=8$.}
    \label{fig:conv_n8}
\end{figure}
\medskip
\begin{figure}[h!]
    \centering
    \ConvPanel{9}
    \caption{Convolution for $n=9$.}
    \label{fig:conv_n9}
\end{figure}

\end{document}
