\makeatletter
\def\input@path{{../styles/}{../../styles/}{../../../styles/}{../}{../../}{../../../}}
\makeatother
\documentclass{ee102_notes}
% macros.tex - Course meta information
\renewcommand{\course}{EE 102}
\renewcommand{\coursetitle}{Signal Processing and Linear Systems}
\renewcommand{\instructor}{Ayush Pandey}
\renewcommand{\semester}{Fall}
\renewcommand{\year}{2025}
\renewcommand{\shorttitle}{Week 1: Introduction to Signals}
% Use \renewcommand to avoid 'already defined' errors

% The following packages can be found on http:\\www.ctan.org
% \usepackage{graphics} % for pdf, bitmapped graphics files
%\usepackage{epsfig} % for postscript graphics files
%\usepackage{mathptmx} % assumes new font selection scheme installed
%\usepackage{times} % assumes new font selection scheme installed
\usepackage{amsmath} % assumes amsmath package installed
\usepackage{amssymb,mathtools}  % assumes amsmath package installed
\usepackage{xcolor}
\usepackage{pgfplots,subcaption}
\usepackage[hidelinks]{hyperref}
\usepackage{verbatim}
\usepackage{graphicx}
\usepackage{listings}
\usepackage{fancyhdr}
% \usepackage{geometry}
\usepackage{siunitx}
\usepackage[most]{tcolorbox}
\usepackage{enumitem}
\usepackage{environ}
% -------- listings (Python) ----------
\lstdefinestyle{py}{
  language=Python,
  basicstyle=\ttfamily\small,
  keywordstyle=\color{blue!60!black}\bfseries,
  commentstyle=\color{green!40!black},
  stringstyle=\color{orange!60!black},
  showstringspaces=false,
  columns=fullflexible,
  frame=single,
  framerule=0.3pt,
  numbers=left,
  numberstyle=\tiny,
  xleftmargin=1em,
  tabsize=2,
  breaklines=true,
}

\usepackage[american]{circuitikz}
\usepackage{tikz}
\usetikzlibrary{arrows.meta,positioning,calc,angles,quotes}
\tikzset{
  >={Latex[length=2.2mm]},
  block/.style={draw, thick, rectangle, minimum height=10mm, minimum width=24mm, align=center},
  gain/.style={block, minimum width=14mm},
  sum/.style={draw, thick, circle, inner sep=0pt, minimum size=6mm},
  conn/.style={-Latex, thick},
}
\usepackage{caption}    
\usepackage{lscape}
\usepackage{soul}
\usepackage{physics}
\usepackage{hyperref}
\hypersetup{
    colorlinks=true,
    linkcolor=blue,
    filecolor=magenta,      
    urlcolor=blue,
    pdftitle={week1_notes},
    pdfpagemode=FullScreen,
}
%\usepackage{float} 

%\usepackage[demo]{graphicx}
\pgfplotsset{compat=1.18}
% \usepgfplotslibrary{fillbetween}

\newsavebox{\measurebox}

\let\proof\relax\let\endproof\relax


\def\abs#1{\left\lvert#1\right\rvert}
\let\proof\relax
\let\endproof\relax
\usepackage{amsthm}
\usepackage{accents}
\usepackage{relsize}
\newcommand{\ubar}[1]{\underaccent{\bar}{#1}}
\newtheorem{theorem}{Theorem}
\newtheorem{corollary}{Corollary}[theorem]
\newtheorem{lemma}{Lemma}
\newtheorem{proposition}{Proposition}
\newtheorem{statement}{Statement}

\theoremstyle{definition}
\newtheorem{definition}{Definition}
 
\theoremstyle{remark}
\newtheorem*{remark}{Remark}
\theoremstyle{remark}
\newtheorem*{claim}{Claim}
\setlength{\parindent}{0cm}
\newenvironment{nalign}{
    \begin{equation}
    \begin{aligned}
}{
    \end{aligned}
    \end{equation}
    \ignorespacesafterend
}

\renewcommand{\releasedate}{September 29, 2025}

\newcommand{\Eblank}{\rule{3cm}{0.4pt}}
\newcommand{\Rankblank}{\rule{3cm}{0.4pt}}

\newcommand{\uof}[1]{u\!\left[#1\right]} % discrete-time unit step

\begin{document}
% \section{In-class activity on convolution}
% ---- knobs you can tweak ----
\def\Alpha{0.5}   % = 1/2
\def\HLen{6}      % h[n] nonzero for n=0..6
\def\Xmin{-2} \def\Xmax{5}    % x[k] axis range
\def\Kmin{-6} \def\Kmax{10}   % h[n-k], y[n] axis range
\def\xscale{0.65cm} \def\yscale{1.5cm}
\def\panelgap{0.9cm}

% ---- helper: empty axes with labels ----
\newcommand{\EmptyAxes}[4]{%
% #1 xmin, #2 xmax, #3 y-label, #4 x-label (usually k)
\begin{tikzpicture}[x=\xscale,y=\yscale]
  \draw[->] (#1-0.6,0) -- (#2+0.6,0) node[below] {#4};
  \draw[->] (0,-0.05) -- (0,1.05) node[left] {#3};
  \foreach \t in {#1,...,#2} \node[below=3pt] at (\t,0) {\scriptsize \t};
\end{tikzpicture}%
}

% ---- (A) BLANK active-learning panel for a given n ----
\newcommand{\BlankConvPanel}[1]{%
\noindent\textbf{Graphically solve for $n=#1$}\\
\EmptyAxes{\Xmin}{\Xmax}{$x[k]$}{$k$}\hspace{\panelgap}%
\EmptyAxes{\Kmin}{\Kmax}{$h[n-k]$}{$k$}\hspace{\panelgap}%
\EmptyAxes{\Kmin}{\Kmax}{$y[n]$}{$n$}\\[8pt]
}

% ---- (B) FILLED panel for n=0 with answer + short solution ----
\newcommand{\FilledPanelNZero}{%
\noindent\textbf{(Solved) Graphically show for $n=0$}\\
% x[k]
\begin{tikzpicture}[x=0.65*\xscale,y=0.65*\yscale]
  \draw[->] (\Xmin-0.6,0) -- (\Xmax+0.6,0) node[below] {$k$};
  \draw[->] (0,-0.05) -- (0,1.15) node[left] {$x[k]$};
  \foreach \t in {\Xmin,...,\Xmax} \node[below=3pt] at (\t,0) {\scriptsize \t};
  % stems: x[0]=x[1]=x[2]=1
  \foreach \k in {0,1,2}{
    \draw[very thick] (\k,0) -- (\k,1);
    \fill (\k,1) circle (1.6pt);
  }
\end{tikzpicture}
\hspace{\panelgap}
% h[n-k] with n=0: nonzero for k in [-6,0], value alpha^{-k}
\begin{tikzpicture}[x=0.65*\xscale,y=0.65*\yscale]
  \draw[->] (\Kmin-0.6,0) -- (\Kmax-4.0,0) node[below] {$k$};
  \draw[->] (0,-0.05) -- (0,1.15) node[left] {$h[n-k]$};
  \foreach \t in {-6,...,2} \node[below=3pt] at (\t,0) {\scriptsize \t};
  \foreach \k in {-6,-5,...,0}{
    \pgfmathparse{pow(\Alpha,-\k)}
    \let\val\pgfmathresult
    \draw[very thick] (\k,0) -- (\k,\val);
    \fill (\k,\val) circle (1.6pt);
  }
\end{tikzpicture}
\hspace{\panelgap}
% y[n] single stem at n=0 of height 1

\begin{tikzpicture}[x=0.65*\xscale,y=0.65*\yscale]
  \draw[->] (\Kmin+3.0,0) -- (\Kmax-4.0,0) node[below] {$n$};
  \draw[->] (0,-0.05) -- (0,1.15) node[above] {$y[0]=\sum_{k} x[k]\,h[0-k]
= x[0]\,h[0]+x[1]\,h[-1]+x[2]\,h[-2]
= 1+0+0=1.$};
  \foreach \t in {-2,...,2} \node[below=3pt] at (\t,0) {\scriptsize \t};
  \draw[very thick] (0,0) -- (0,1);
  \fill (0,1) circle (1.8pt);
\end{tikzpicture}\\[8pt]
}

% ---- (C) Generate the 10 requested panels ----
% Tip: Put five per page using manual page breaks or minipages.
% Page 1 (example):
\thispagestyle{fancy}
% add header: EE 102 activity 
\lhead{NAME: \underline{\hspace{5cm}}}
\rhead{EE 102: In-class activity}
\section*{Visualize convolution}
\BlankConvPanel{-1}
\hrule
\FilledPanelNZero
\hrule
\BlankConvPanel{1}
% \hrule[4pt]
\newpage
\BlankConvPanel{2}
\hrule
\BlankConvPanel{3}
\hrule
\BlankConvPanel{4}
% \hrule[2pt]
\newpage

\BlankConvPanel{5}
\hrule
\BlankConvPanel{6}
\hrule
\BlankConvPanel{7}
\end{document}